% Please make sure you insert your
% data according to the instructions in PoSauthmanual.pdf
\documentclass[a4paper,11pt]{article}
\usepackage{pos}
\usepackage{sidecap}    
\sidecaptionvpos{figure}{m}    
\sidecaptionvpos{table}{m}    
\usepackage{cleveref}    
\usepackage{enumitem}
\setitemize{itemsep=0pt,parsep=1pt,topsep=1pt,itemindent=0pt,leftmargin=9pt}    
\setenumerate{itemsep=0pt,parsep=1pt,topsep=1pt,itemindent=10pt,leftmargin=9pt}    
\usepackage{natbib}    
\setlength{\bibsep}{2pt}    
\setlength{\parindent}{0pt}    

\usepackage{listings}

\definecolor{codegreen}{rgb}{0,0.6,0}
\definecolor{codegray}{rgb}{0.5,0.5,0.5}
\definecolor{codepurple}{rgb}{0.58,0,0.82}
\definecolor{backcolour}{rgb}{0.95,0.95,0.92}

\lstdefinestyle{mystyle}{
    backgroundcolor=\color{backcolour},   
    commentstyle=\color{codegreen},
    keywordstyle=\color{magenta},
    numberstyle=\tiny\color{codegray},
    stringstyle=\color{codepurple},
    basicstyle=\ttfamily\tiny,
    breakatwhitespace=false,
    breaklines=true, 
    captionpos=b,
    keepspaces=true,
    numbers=left,
    numbersep=5pt,
    showspaces=false,
    showstringspaces=false,
    showtabs=false,
    tabsize=2
}

\lstset{style=mystyle}

\usepackage{algorithm}
\usepackage{algorithmicx}
\usepackage{algpseudocode}

\usepackage[font={small,it},labelfont=bf,tableposition=top]{caption}

\input{commands.tex}

\title{Status of the ETMC ensemble generation effort}
%% \ShortTitle{Short Title for header}

\author[a,b]{C.~Alexandrou}
\author[b]{S.~Bacchio}
\author[c]{J.~Finkenrath}
\author[d]{R.~Frezzotti}
\author*[e]{M.~Garofalo}
\author*[e]{B.~Kostrzewa}
\author[b]{G.~Koutsou}
\author[f]{S.~Romiti}
\author[e]{A.~Sen}
\author[e]{C.~Urbach}
\author[f]{U.~Wenger}

\affiliation[a]{Department of Physics, University of Cyprus, 20536 Nicosia, Cyprus}
\affiliation[b]{Computation-based Science and Technology Research Center, The Cyprus Institute, 2121 Nicosia, Cyprus}
\affiliation[c]{Theoretical Physics Department, CERN 1211 Geneva 23, Switzerland}
\affiliation[d]{Dipartimento di Fisica and INFN, Universit{\`a} di Roma ``Tor Vergata'', I-00133}
\affiliation[e]{Helmholtz-Institut für Strahlen und Kernphysik (Theory), Rheinische Friedrich-Wilhelms-Universität Bonn, Nussallee 14-16, 53115 Bonn, Germany}
\affiliation[f]{Institute for Theoretical Physics, Albert Einstein Center for Fundamental Physics, University of Bern, CH-3012 Bern, Switzerland}

%\emailAdd{alexandrou.constantia@ucy.ac.cy}
%\emailAdd{s.bacchio@cyi.ac.cy}
%\emailAdd{j.finkenrath@cern.ch}
%\emailAdd{roberto.frezzotti@roma2.infn.it}
\emailAdd{garofalo@hiskp.uni-bonn.de}
\emailAdd{kostrzewa@hiskp.uni-bonn.de}
%\emailAdd{g.koutsou@cyi.ac.cy}
%\emailAdd{simone.romiti@unibe.ch}
%\emailAdd{sen@hiskp.uni-bonn.de}
%\emailAdd{urbach@hiskp.uni-bonn.de}
%\emailAdd{urs.wenger@unibe.ch}

\abstract{We present a status report on the ETMC ensemble generation effort toward controlled continuum and infinite volume extrapolations for a variety of physical observables through simulations employing $N_f = 2+1+1$ Wilson clover twisted mass fermions at physical quark masses using five lattice spacings.
We further given an update on the status of the tmLQCD software suite.
Through extensions of the QUDA lattice QCD library and a corresponding interface in tmLQCD, we are able to offload a significant portion of our HMC to GPUs, enabling efficient simulations on the current generation of heterogeneous machines.
\vspace{1cm}
\begin{center}
  \includegraphics[width=0.30\linewidth]{plots/Logo_ETMC_RGB}
\end{center}
}

\FullConference{The 41st International Symposium on Lattice Field Theory (LATTICE2024)\\
 28 July - 3 August 2024\\
Liverpool, UK\\}

%% \tableofcontents

\begin{document}
\maketitle


\section{Introduction}


\begin{figure}
  \includegraphics[width=0.5\linewidth,page=2]{plots/quda_speedup}
  \includegraphics[width=0.5\linewidth,page=1]{plots/quda_speedup}
  \caption{\textbf{Left:} Time per unit length trajectory of tmLQCD + QUDA (green) for a $64^3 \cdot 128$ ensemble at the physical point compared to the machine it was originally generated on (purple) using tmLQCD + DD-$\alpha$AMG~\cite{Alexandrou:2016izb}. The different speedups correspond to increasing levels of QUDA offloading. \textbf{Right:} The same kind of comparison between tmLQCD + QUDA and tmLQCD + DD-$\alpha$AMG + QPhiX~\cite{QPhiX,Schrock:2015gik,QPhiX-github} running a $112^3 \cdot 224$ ensemble at the physical point on LUMI-G and Frontera, respectively.}
  \label{fig:quda_speedup}
\end{figure}

\section{Lattice Setup}
    The $N_f =2+1+1$ path integral for twisted mass Wilson clover fermions \cite{Frezzotti:2003ni,Frezzotti:2004wz,Sheikholeslami:1985ij} is
    \begin{equation*}
      Z= \int \mathcal{D}U \mathcal{D}\chi \mathcal{D}\bar\chi \,e^{-S_\mathrm{gauge}-\bar \chi D_\ell\chi - \bar \chi D_h \chi } \,,
    \end{equation*}
    where $D_\ell$ is the Dirac operator for a doublet of light mass-degenerate quarks and $D_h$ is the Dirac operator for a non-degenerate doublet corresponding to the strange and charm contributon:
    \begin{equation*}
        D_\ell = (\Dsw[U] + m_0)\ 1_f + i \mu_\ell\gamma_5\tau^3_f\, ,\quad\quad
        D_h = (\Dsw[U] + m_0)\ 1_f + i\bar\mu\gamma_5\tau^3_f - \bar\epsilon \tau^1_f
    \end{equation*}
    \begin{equation*}
        Q = \gamma_5 D \underset{\text{e/o precon}}{\rightarrow} \hat{Q} \underset{\text{Hasenbusch}}{\rightarrow} \hat{W}(\pm\rho) = \hat{Q} \pm i\rho
    \end{equation*}
    where $\Dsw$ is the Wilson clover operator while $m_0$, $\mu_\ell$, $\mubar$ and $\epsbar$ are the various untwisted and twisted mass parameters.
    We employ Hasenbusch mass-preconditioning \cite{Hasenbusch:2001ne} to split the light quark determinant and rational HMC \cite{Clark:2006fx} with split partial fractions in the non-degenerate one. Even-odd preconditioning is used throughout.

The various inversions are performed using the most appropriate solver for each monomial.
\vspace{1cm}
\begin{tabular}{p{0.3\linewidth}p{0.7\linewidth}}
  \centering $\frac{1}{\What_{+}(\rho_t) \What_{-}(\rho_t)}$ & double-half mixed-precision CG \\
  \centering $\What_{-}(\rho_t) \frac{1}{\What_{+}(\rho_b) \What_{-}(\rho_b)} \What_{+}(\rho_t)$ & double-half mixed precision CG or multigrid-preconditoned GCR depending on $\rho_b$ \\
  \centering $\prod_{i=n_\ell}^{n_k} \frac{ \Qhat^2_h + a_{2i-1} }{ \Qhat^2_h + a_{2i} }$ & single precision multi-shift CG with double-half precision shift-by-shift refinement 
\end{tabular}
Integrating fermion variables and using Even/odd Preconditioning
\begin{flalign*}
    Z= \int DU  \,e^{-S_{gauge} }\det{\left(M_{ee}^+M_{ee}^-\right)}
    \det{(\hat Q_{+}\hat Q_{-})}
    \det{(\hat Q_h)} \det{(M_{ee}^{h})} &  &
\end{flalign*}
with the degenerate operator written in terms of the even/odd part of the operator
\begin{flalign*}
    \hQpm = \gamma_5 \left[  \Moo   - \Moe \Mee^{-1} \Meo \right] \,,\quad \quad
    \Mee =1 + 2\kappa c_{SW} T_{ee} + i\tilde\mu\gamma_5 \,, \quad \quad M_{deg} =\gamma_5 \begin{pmatrix}
                                                                                               M_{ee} & M_{eo} \\
                                                                                               M_{oe} & M_{oo} \\
                                                                                           \end{pmatrix} &  &
\end{flalign*}
while the non-degenerate operator is a two-flavour operators
\begin{flalign*}
    \hat	Q_h = \gamma_5 \left[ ( \Moo^h + i \mu_\sigma \gamma_5 \tau^3 - \mu_\delta \tau^1 ) - \Moe^h (\Mee^h + i \mu_\sigma \gamma_5 \tau^3 - \mu_\delta \tau^1 )^{-1} \Meo^h \right] &  &
\end{flalign*}

Applaying Hasenbusch trick \cite{Hasenbusch:2001ne} for the degenerate and Rational HMC \cite{Clark:2006fx} for the non degenerate operator
we get
\begin{multline*}
    Z= \int DU  \,e^{-S_{gauge} }\det{\left(M_{ee}^+M_{ee}^-\right)}
    \det{(\hat W_{+}\hat W_{-})}	\det{(\hat W_{+}^{-1}\hat Q_+\hat Q_-\hat W_{-}^{-1})}...\,\\
    \det{(M_{ee}^{h})}
    {\det{(  r_1(\hat Q_h) )}\det{(  r_2(\hat Q_h) )}...\,\det{(|\hat Q_h|{\cal R}(\hat Q_h)}}
\end{multline*}
with the operator
$\hWpm(\rho) = \hQpm \pm i \rho$ s.t. $\hWp(\rho) \hWm(\rho) = \hQp \hQm +\rho^2$ such that the clover inverse $\rho$-independent.
The rational approximation is
\begin{equation*}
    \mathcal{R}(\hat Q_h^2) = \prod_{i=1}^{N} \frac{\hat Q^2_h + a_{2i-1}}{\hat Q^2_h + a_{2i}}=\prod_{i=1}^{N} r_i(\hat Q_h) \approx \frac{1}{\sqrt{\hat Q_h^2}}
\end{equation*}
with $N \approx 10$, with $\mathcal{R}$ split across 2-3 monomials $r_i(\hat Q_h)$ on 2-3 timescales (usually 3).
Exponentiating the determinants we get the action used in the HMC \cite{Duane:1987de}
\begin{flalign*}
    Z= \int DU D\phi D\phi^*  \,e^{-S_{gauge} } e^{-S_{det}}
    e^{-S_{PF}}  e^{-S_{PF}^1}...\,
    {e^{-S_{det}^h}}
    {e^{-S_{det}^h}}
    e^{-S_{PF}^{1h}} e^{-S_{PF}^{2h}}...\,
    e^{-S_{corr}}\,.
\end{flalign*}
The corresponding equation of motion to be integrated in the HMC are
\begin{align*}
    \partial_t U   & =\Pi                \\
    \partial_t \Pi & =-\delta S_{gauge}-
    \delta S_{det} - \delta S_{PF} - \delta S_{PF}^1 -...-
    \delta S^h_{det} -\delta S^{1h}_{PF} - \delta S^{2h}_{PF}-...-
    \delta S_{corr}\,.
\end{align*}

\section{Overview of Current Ensembles}

\begin{table}
  \small
  \begin{tabular}{lrrlllllll}
    \textbf{ensemble} & $L/a$ & $T/a$ & $\sim a$ [fm] & $M_\pi$ [MeV] & $\beta$ & $\csw$ & $a\mu_\ell$ & $a\mu_\sigma$ & $a\mu_\delta$ \\ 
    \hline 
    cA211.12.48   & 48    & 96    & 0.091         & 174   & 1.726   & 1.7400 & 0.00120    & --           & -- \\
    cA211.15.48   & 48    & 96    & --            & 194   & --      & --     & 0.00150    & --           & -- \\
    cA211.15.64   & 64    & 128   & --            & --    & --      & --     & --         & --           & -- \\
    cA211.30.32   & 30    & 64    & --            & 272   & --      & --     & 0.00300    & --           & -- \\
    cA211.40.24   & 24    & 48    & --            & 315   & --      & --     & 0.00400    & --           & -- \\
    cA211.53.24   & 24    & 48    & --            & 360   & --      & --     & 0.00530    & --           & -- \\ \hline
    cAp211.085.48 & 48    & 96    & 0.087         & 145   & 1.745   & 1.7112 & 0.00085    & --           & -- \\
    cAp211.085.56 & 48    & 96    & --            & --    & --      & --     & --         & --           & -- \\ \hline
    cB211.072.96  & 96    & 192   & 0.080         & 140   & 1.778   & 1.6900 & 0.00072    & --           & -- \\
    cB211.072.64  & 64    & 128   & --            & --    & --      & --     & --         & --           & -- \\
    cB211.14.64   & 64    & 128   & --            & 194   & --      & --     & 0.00140    & --           & -- \\  
    cB211.14.48   & 48    & 96    & --            & --    & --      & --     & 0.00140    & --           & -- \\ 
    cB211.25.48   & 48    & 96    & --            & 260   & --      & --     & 0.00250    & --           & -- \\
    cB211.25.32   & 32    & 64    & --            & --    & --      & --     & --         & --           & -- \\
    cB211.25.24   & 24    & 48    & --            & --    & --      & --     & --         & --           & -- \\ \hline 
    cC211.06.112  & 112   & 224   & 0.068         & 137   & 1.836   & 1.6452 & 0.00060    & --           & -- \\
    cC211.06.80   & 80    & 160   & --            & --    & --      & --     & --         & --           & -- \\
    cC211.20.48   & 48    & 96    & --            & --    & --      & --     & 0.00200    & --           & -- \\ \hline
    cD211.054.128 & 128   & 256   & 0.057         & 141   & 1.900   & 1.6112 & 0.00054    & --           & -- \\
    cD211.054.96  & 96    & 192   & --            & --    & --      & --     & --         & --           & -- \\ \hline
    cE211.044.112 & 112   & 224   & 0.049         & 136   & 1.960   & 1.6792 & 0.00044    & --           & -- \\ \hline
  \end{tabular}
\end{table}

\begin{figure}
  \includegraphics[height=7cm]{plots/ensembles_asquared_mpi}\hfill
  \includegraphics[height=7cm]{plots/ensembles_L_vs_mpi}
  \caption{Overview of all $N_f=2+1+1$ ETMC Wilson clover twisted mass ensembles as a function of the pion mass, the lattice spacing and $M_\pi \cdot L$. Points are slightly displaced either horizontally or vertically for visibility reasons.}
  \label{fig:ensemble_overview}
\end{figure}

\begin{SCfigure}
  \includegraphics[width=0.5\linewidth]{plots/ensembles_phys_point}
  \caption{Overview of ETMC ensembles with $M_\pi \lesssim 147$ MeV as a function of the lattice spacing squared and the lattice volume in fermi. The different shaded regions indicate $M_\pi \cdot L$.}
  \label{fig:ensembles_phys_point}
\end{SCfigure}

\begin{SCfigure}
  \includegraphics[width=\linewidth]{plots/amu_s_W_a4}
\end{SCfigure}

\begin{figure}
  \includegraphics[width=0.49\linewidth,page=1]{../data/gf_observables/gf_observables_md_histories}
  \hfill
  \includegraphics[width=0.49\linewidth,page=2]{../data/gf_observables/gf_observables_md_histories}
  \caption{\textbf{Left:} Molecular dynamics histories for different ensembles close to or ar the physical point of the flow-time derivative of the gradient-flowed energy density, interpolated to the flow time where the ensemble average $t \frac{t}{dt} \lbrace t^2 \langle E(t) \rangle \rbrace = W(t)|_{t = w_0^2} = 0.3$. Only the first 2200 unit-length molecular dynamics units are shown in the histories whereas the histograms in the right sub-panels represent the full data sets. \textbf{Right:} Similar histories for the gradient-flowed topological charge in the clover definition evaluated at a flow time $t = w_0^2$.}
\end{figure}


\section{Acknowledgements}
{
\small
We would like to thank the QUDA developers for their tremendous work as well as the many pleasant and productive interactions during this and previous efforts. 
We thank the ETMC for the most enjoyable collaboration. 
For part of this work. S.B. and J.F. were supported by the H2020 project PRACE 6-IP (grant agreement No. 82376) and the EuroCC project (grant agreement No. 951740).
This project was funded by the Deutsche Forschungsgemeinschaft (DFG, German Research Foundation) as part of the CRC 1639 NuMeriQS – project no. 511713970. 
This work was supported by the Deutsche Forschungsgemeinschaft (DFG, German Research Foundation) and the NSFC through the funds provided to the Sino-German Collaborative Research Center CRC 110 “Symmetries and the Emergence of Structure in QCD” (DFG Project-ID 196253076 - TRR 110, NSFC Grant No. 12070131001).
We acknowledge support by the European Joint Doctorate program STIMULATE grant agreement No. 765048.
We acknowledge support from projects NextQCD (EXCELLENCE/0918/0129) and "3D-Nucleon" (EXCELLENCE/0421/0043), co-funded by the European Regional Development Fund and the Republic of Cyprus through the Research and Innovation Foundation. 
We further acknowledge: The Gauss Centre for Supercomputing e.V. for funding this project through computing time on the GCS supercomputers SuperMUC and SuperMUC-NG at Leibniz Supercomputing Center as well as JUWELS Booster~\cite{JUWELS,BOOSTER} at the Jülich Supercomputing Centre.
PRACE for awarding access to HAWK at HLRS within the project with Id Acid 4886. 
The Swiss National Supercomputing Centre (CSCS) and the EuroHPC Joint Undertaking for awarding access to the LUMI supercomputer, owned by the EuroHPC JU, hosted by CSC and the LUMI consortium through the Chronos programme under project IDs CH17-CSCS-CYP and CH21-CSCS-UNIBE as well as EuroHPC under project ID EHPC-REG-2021R0095.
The Texas Advanced Computing Center (TACC) at The University of Texas at Austin for providing HPC resources on Frontera~\cite{FRONTERA} (Project ID PHY21001).
The EuroHPC Joint Undertaking for awarding access to the Luxembourg national supercomputer MeluXina.
The University of Bonn for access to the Bender and Marvin clusters as well as support by the HRZ-HPC team.
}

\bibliographystyle{JHEP}
{\footnotesize
\bibliography{bibliography}
}

\end{document}
